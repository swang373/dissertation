\chapter{CONCLUSIONS} \label{conclusions}

A search for the Standard Model Higgs boson produced in association with a weak vector boson and decaying into a bottom-antibottom quark pair (\bb) has been presented for the \VHbb\ decay channels \ZnnHbb, \WenHbb, \WmnHbb, \ZeeHbb, and \ZmmHbb. The search is performed with a dataset corresponding to an integrated luminosity of 41.3 \invfb\ at a center-of-mass energy of $\sqrt{s} = 13\ \TeV$ recorded by the CMS experiment at the LHC during Run 2 in 2017. An excess of events in data over the expected background is observed with a significance of 3.3 standard deviations, as compared to an expected significance of 3.1 standard deviations under the Standard Model prediction for a Higgs boson of mass \massH\ = 125.09 \GeV decaying to bottom quarks. The corresponding signal strength of this excess relative to the Standard Model Higgs boson is measured to be $\mu = 1.08_{-0.33}^{+0.35}$, which suggests compatibility with the Standard Model.

A grand combination of this result is performed with previous measurements made by the CMS collaboration of the associated vector boson, gluon fusion, vector boson fusion, and associated top quark production modes for \Htobb. This combination achieves an observed significance of 5.6 standard deviations, with an expected significance of 5.5 standard deviations, and measures an overall signal strength of $\mu = 1.04 \pm 0.20$ for the \Htobb\ decay. This constitutes the observation of the \Htobb\ decay by the CMS collaboration, with no clear indication of physics beyond the Standard Model.

Although the \Htobb\ decay has been observed, the continued study of this decay offers an opportunity to check for anomalous Higgs' Yukawa couplings and, given it has the largest branching fraction of all Higgs' decays, to improve the constraints placed on the total decay width of the Higgs boson. The focus of future analyses will then prioritize improving the precision of the measurement of the \VHbb\ production cross section and the \Htobb\ branching fraction. By refining the strategy adopted for the dijet invariant mass analysis, where a multivariate classifier is evaluated in such a way that its output is decorrelated from the fitted distribution, the effects of shape systematics could be reduced as the dijet invariant mass distributions of the background processes are typically flat. Similar techniques that warrant further investigation include a method of rescaling the input features to decorrelate a multivariate classifier from specific variables that has been successfully used by the \Htomm\ analysis\cite{HTOMUMU} and an adversarial training method that has been proposed to decorrelate jet tagging algorithms from the jet mass\cite{ADVTRAIN}.

Future analyses of the \VHbb\ decay into Run 3 and beyond face increased luminosities which may cause the dijet invariant mass resolution and \qrkb-tagging performance to degrade due to increased pileup. More robust methods of pileup removal such as the pileup per particle identification (PUPPI)\cite{PUPPI} algorithm may be required under these harsher conditions. Future analyses performed at higher center-of-mass energies may also benefit from a dedicated jet substructure\cite{JETSUB} analysis, given that the bottom quarks from the Higgs decay become highly collimated at very high boosts. While typical jet algorithms may be unable to resolve the merged \qrkb-jets, jet substructure techniques such as those used by the search for \Htobb\ using the gluon fusion production mode\cite{ggHbb} are capable of identifying jet constituents and their flavor. Novel applications of machine learning in the field of high energy physics towards jet flavor tagging, such as the DeepJet\cite{DEEPJET} algorithm, and discovery significance optimization\cite{OPTSIG} may improve the sensitivity of the analysis.
