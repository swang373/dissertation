% user defined commands %
% Here is where you define optional commands such as macros, new commands,
% and new environments to be used in your paper

% optional command to prevent a word from breaking across a line %
\hyphenchar\font=-1


% Commands to produce proper bullet list
\newlength{\widthOfItem}
\let\Itemize=\itemize
\let\endItemize=\enditemize
\renewenvironment{itemize}{%
	\begin{Itemize}
		\setlength{\itemsep}{0.5\baselineskip}
		\setlength{\labelwidth}{2em}
		\setlength{\listparindent}{.32in}%
		\setlength{\leftmargin}{.32in}
		\setlength{\rightmargin}{0in}
		\settowidth{\widthOfItem}{\labelitemi}
		\setlength{\labelsep}{\leftmargin-\widthOfItem}
		\renewcommand{\labelitemii}{--}
		\singlespacing}{%
	\end{Itemize}}

% shortcut for setting up inserting \prime command in mathmode to avoid errors %
\newcommand{\p}{^{\prime}}

% shortcuts for prime color text
\newcommand{\red}{\textcolor[rgb]{1.00,0.00,0.00}}
\newcommand{\green}{\textcolor[rgb]{0.00,1.00,0.00}}
\newcommand{\blue}{\textcolor[rgb]{0.00,0.00,1.00}}

% Shorcut commands for mathmatical formulas %

\newcommand{\latex}{\LaTeX 2\ensuremath{\epsilon}}

% THEOREM Environments ---------------------------------------------------
%These environments are provided as a convenience - feel free to modify if needed

\newtheorem{theorem}{Theorem}[chapter]%To link the theorem to each chapter uncomment the chapter option
\newtheorem{lemma}{Lemma}%[theorem]% To link each lemma to a theorem uncomment the theorem option
\newtheorem{corollary}{Corollary}%[theorem]% To link each corollary to a theorem uncomment the theorem option
% to link a corollary to a chapter change the theorem option to chapter
\newtheorem{definition}{Definition}%[chapter] %the same is true for both definitions and assumptions
\newtheorem{assumption}{Assumption}%[chapter] %
\newtheorem{proposition}{Proposition}[chapter]
\newtheorem{algorithm}{Algorithm}[chapter]

%%%%%%%%%%%%%%%%%%%%%%%%%%%%%
% Particle Physics Commands %
%%%%%%%%%%%%%%%%%%%%%%%%%%%%%

% Units
\newcommand{\invfb}{\ensuremath{\mathrm{fb^{-1}}}}
\newcommand{\GeV}{\ensuremath{\mathrm{GeV}}}
\newcommand{\TeV}{\ensuremath{\mathrm{TeV}}}

% Particle Symbols
\newcommand{\syme}{\ensuremath{\mathrm{e}}}
\newcommand{\symb}{\ensuremath{\mathrm{b}}}
\newcommand{\symH}{\ensuremath{\mathrm{H}}}
\newcommand{\symV}{\ensuremath{\mathrm{V}}}
\newcommand{\symW}{\ensuremath{\mathrm{W}}}
\newcommand{\symZ}{\ensuremath{\mathrm{Z}}}

% Shorthand
\newcommand{\VH}{\ensuremath{\mathrm{VH}}}
\newcommand{\bb}{\ensuremath{\symb\bar{\symb}}}
\newcommand{\VHbb}{\ensuremath{\VH(\bb)}}

% Decay Formulae
\newcommand{\Htobb}{\ensuremath{\symH\rightarrow\symb\bar{\symb}}}

% VHbb Decay Channels
\newcommand{\Znn}{\ensuremath{\symZ(\nu\bar{\nu})}}
\newcommand{\ZnnH}{\ensuremath{\symZ(\nu\bar{\nu})\symH}}
\newcommand{\ZnnHbb}{\ensuremath{\symZ(\nu\bar{\nu})\symH(\bb)}}

\newcommand{\Wln}{\ensuremath{\symW(\ell\nu)}}
\newcommand{\WlnH}{\ensuremath{\symW(\ell\nu)\symH}}
\newcommand{\WlnHbb}{\ensuremath{\symW(\ell\nu)\symH(\bb)}}

\newcommand{\Wen}{\ensuremath{\symW(\syme\nu)}}
\newcommand{\WenH}{\ensuremath{\symW(\syme\nu)\symH}}
\newcommand{\WenHbb}{\ensuremath{\symW(\syme\nu)\symH(\bb)}}

\newcommand{\Wmn}{\ensuremath{\symW(\mu\nu)}}
\newcommand{\WmnH}{\ensuremath{\symW(\mu\nu)\symH}}
\newcommand{\WmnHbb}{\ensuremath{\symW(\mu\nu)\symH(\bb)}}

\newcommand{\Zll}{\ensuremath{\symZ(\ell\bar{\ell})}}
\newcommand{\ZllH}{\ensuremath{\symZ(\ell\bar{\ell})\symH}}
\newcommand{\ZllHbb}{\ensuremath{\symZ(\ell\bar{\ell})\symH(\bb)}}

\newcommand{\Zee}{\ensuremath{\symZ(\syme\bar{\syme})}}
\newcommand{\ZeeH}{\ensuremath{\symZ(\syme\bar{\syme})\symH}}
\newcommand{\ZeeHbb}{\ensuremath{\symZ(\syme\bar{\syme})\symH(\bb)}}

\newcommand{\Zmm}{\ensuremath{\symZ(\mu\bar{\mu})}}
\newcommand{\ZmmH}{\ensuremath{\symZ(\mu\bar{\mu})\symH}}
\newcommand{\ZmmHbb}{\ensuremath{\symZ(\mu\bar{\mu})\symH(\bb)}}

% Kinematic Variables and Other Quantities
\newcommand{\sqrts}{\ensuremath{\sqrt{s}}}
\newcommand{\massH}{\ensuremath{m_{\symH}}}
\newcommand{\pT}{\ensuremath{p_{\mathrm{T}}}}


%These were some user commands I've run across that I thought some might want to incorporate into their work
%\newcommand{\bdm}{
 %   \begin{displaymath}}

%\newcommand{\edm}{
%    \end{displaymath}}

%\newcommand{\be}{
%    \begin{equation}}

%\newcommand{\ee}{
%    \end{equation}}

%\newcommand{\bea}{
 %   \begin{eqnarray}}

%\newcommand{\eea}{
%    \end{eqnarray}}
