% user defined commands %
% Here is where you define optional commands such as macros, new commands,
% and new environments to be used in your paper

% optional command to prevent a word from breaking across a line %
\hyphenchar\font=-1


% Commands to produce proper bullet list
\newlength{\widthOfItem}
\let\Itemize=\itemize
\let\endItemize=\enditemize
\renewenvironment{itemize}{%
	\begin{Itemize}
		\setlength{\itemsep}{0.5\baselineskip}
		\setlength{\labelwidth}{2em}
		\setlength{\listparindent}{.32in}%
		\setlength{\leftmargin}{.32in}
		\setlength{\rightmargin}{0in}
		\settowidth{\widthOfItem}{\labelitemi}
		\setlength{\labelsep}{\leftmargin-\widthOfItem}
		\renewcommand{\labelitemii}{--}
		\singlespacing}{%
	\end{Itemize}}

% shortcut for setting up inserting \prime command in mathmode to avoid errors %
\newcommand{\p}{^{\prime}}

% shortcuts for prime color text
\newcommand{\red}{\textcolor[rgb]{1.00,0.00,0.00}}
\newcommand{\green}{\textcolor[rgb]{0.00,1.00,0.00}}
\newcommand{\blue}{\textcolor[rgb]{0.00,0.00,1.00}}

% Shorcut commands for mathmatical formulas %

\newcommand{\latex}{\LaTeX 2\ensuremath{\epsilon}}

% THEOREM Environments ---------------------------------------------------
%These environments are provided as a convenience - feel free to modify if needed

\newtheorem{theorem}{Theorem}[chapter]%To link the theorem to each chapter uncomment the chapter option
\newtheorem{lemma}{Lemma}%[theorem]% To link each lemma to a theorem uncomment the theorem option
\newtheorem{corollary}{Corollary}%[theorem]% To link each corollary to a theorem uncomment the theorem option
% to link a corollary to a chapter change the theorem option to chapter
\newtheorem{definition}{Definition}%[chapter] %the same is true for both definitions and assumptions
\newtheorem{assumption}{Assumption}%[chapter] %
\newtheorem{proposition}{Proposition}[chapter]
\newtheorem{algorithm}{Algorithm}[chapter]

%%%%%%%%%%%%%%%%%%%%%%%%%%%%%
% Particle Physics Commands %
%%%%%%%%%%%%%%%%%%%%%%%%%%%%%

% Mathematical Symbols
\newcommand{\symEM}{\ensuremath{\mathrm{U(1)}}}
\newcommand{\symWEAK}{\ensuremath{\mathrm{SU(2)}}}
\newcommand{\symSTRONG}{\ensuremath{\mathrm{SU(3)}}}
\newcommand{\symSM}{\ensuremath{\symSTRONG \times \symWEAK \times \symEM}}

% Units
\newcommand{\pb}{\ensuremath{\mathrm{pb}}}
\newcommand{\fb}{\ensuremath{\mathrm{fb}}}
\newcommand{\invpb}{\ensuremath{\mathrm{pb^{-1}}}}
\newcommand{\invfb}{\ensuremath{\mathrm{fb^{-1}}}}
\newcommand{\MeV}{\ensuremath{\mathrm{MeV}}}
\newcommand{\GeV}{\ensuremath{\mathrm{GeV}}}
\newcommand{\TeV}{\ensuremath{\mathrm{TeV}}}

% Particle Symbols
\newcommand{\lepe}{\ensuremath{e}}
\newcommand{\lepm}{\ensuremath{\mu}}
\newcommand{\lept}{\ensuremath{\tau}}
\newcommand{\lepne}{\ensuremath{\nu_{e}}}
\newcommand{\lepnm}{\ensuremath{\nu_{\mu}}}
\newcommand{\lepnt}{\ensuremath{\nu_{\tau}}}

\newcommand{\lepebar}{\ensuremath{\bar{e}}}
\newcommand{\lepmbar}{\ensuremath{\bar{\mu}}}
\newcommand{\leptbar}{\ensuremath{\bar{\tau}}}
\newcommand{\lepnebar}{\ensuremath{\bar{\nu}_{e}}}
\newcommand{\lepnmbar}{\ensuremath{\bar{\nu}_{\mu}}}
\newcommand{\lepntbar}{\ensuremath{\bar{\nu}_{\tau}}}

\newcommand{\qrku}{\ensuremath{u}}
\newcommand{\qrkd}{\ensuremath{d}}
\newcommand{\qrkc}{\ensuremath{c}}
\newcommand{\qrks}{\ensuremath{s}}
\newcommand{\qrkb}{\ensuremath{b}}
\newcommand{\qrkt}{\ensuremath{t}}

\newcommand{\qrkubar}{\ensuremath{\bar{u}}}
\newcommand{\qrkdbar}{\ensuremath{\bar{d}}}
\newcommand{\qrkcbar}{\ensuremath{\bar{c}}}
\newcommand{\qrksbar}{\ensuremath{\bar{s}}}
\newcommand{\qrkbbar}{\ensuremath{\bar{b}}}
\newcommand{\qrktbar}{\ensuremath{\bar{t}}}

\newcommand{\bosg}{\ensuremath{\gamma}}
\newcommand{\bosV}{\ensuremath{V}}
\newcommand{\bosW}{\ensuremath{W}}
\newcommand{\bosWp}{\ensuremath{W^{+}}}
\newcommand{\bosWm}{\ensuremath{W^{-}}}
\newcommand{\bosZ}{\ensuremath{Z}}
\newcommand{\bosZn}{\ensuremath{Z^{0}}}
\newcommand{\bosgln}{\ensuremath{g}}
\newcommand{\bosH}{\ensuremath{H}}

% Shorthand
\newcommand{\VH}{\ensuremath{VH}}
\newcommand{\bb}{\ensuremath{\qrkb\bar{\qrkb}}}
\newcommand{\Hbb}{\ensuremath{\bosH(\bb)}}
\newcommand{\VHbb}{\ensuremath{\VH(\bb)}}
\newcommand{\Wlight}{\ensuremath{\bosW+\qrku\qrkd\qrkc\qrks\bosgln}}
\newcommand{\Wb}{\ensuremath{\bosW+\qrkb}}
\newcommand{\Wbb}{\ensuremath{\bosW+\qrkb\qrkbbar}}
\newcommand{\Zlight}{\ensuremath{\bosZ+\qrku\qrkd\qrkc\qrks\bosgln}}
\newcommand{\Zb}{\ensuremath{\bosZ+\qrkb}}
\newcommand{\Zbb}{\ensuremath{\bosZ+\qrkb\qrkbbar}}

% Decay Formulae
\newcommand{\Htobb}{\ensuremath{\bosH\rightarrow\qrkb\bar{\qrkb}}}

% VHbb Decay Channels
\newcommand{\Znn}{\ensuremath{\bosZ(\nu\bar{\nu})}}
\newcommand{\ZnnH}{\ensuremath{\bosZ(\nu\bar{\nu})\bosH}}
\newcommand{\ZnnHbb}{\ensuremath{\bosZ(\nu\bar{\nu})\bosH(\bb)}}

\newcommand{\Wln}{\ensuremath{\bosW(\ell\nu)}}
\newcommand{\WlnH}{\ensuremath{\bosW(\ell\nu)\bosH}}
\newcommand{\WlnHbb}{\ensuremath{\bosW(\ell\nu)\bosH(\bb)}}

\newcommand{\Wen}{\ensuremath{\bosW(\lepe\nu)}}
\newcommand{\WenH}{\ensuremath{\bosW(\lepe\nu)\bosH}}
\newcommand{\WenHbb}{\ensuremath{\bosW(\lepe\nu)\bosH(\bb)}}

\newcommand{\Wmn}{\ensuremath{\bosW(\lepm\nu)}}
\newcommand{\WmnH}{\ensuremath{\bosW(\lepm\nu)\bosH}}
\newcommand{\WmnHbb}{\ensuremath{\bosW(\lepm\nu)\bosH(\bb)}}

\newcommand{\Zll}{\ensuremath{\bosZ(\ell\bar{\ell})}}
\newcommand{\ZllH}{\ensuremath{\bosZ(\ell\bar{\ell})\bosH}}
\newcommand{\ZllHbb}{\ensuremath{\bosZ(\ell\bar{\ell})\bosH(\bb)}}

\newcommand{\Zee}{\ensuremath{\bosZ(\lepe\bar{\lepe})}}
\newcommand{\ZeeH}{\ensuremath{\bosZ(\lepe\bar{\lepe})\bosH}}
\newcommand{\ZeeHbb}{\ensuremath{\bosZ(\lepe\bar{\lepe})\bosH(\bb)}}

\newcommand{\Zmm}{\ensuremath{\bosZ(\lepm\bar{\lepm})}}
\newcommand{\ZmmH}{\ensuremath{\bosZ(\lepm\bar{\lepm})\bosH}}
\newcommand{\ZmmHbb}{\ensuremath{\bosZ(\lepm\bar{\lepm})\bosH(\bb)}}

% Kinematic Variables and Other Quantities
\newcommand{\sqrts}{\ensuremath{\sqrt{s}}}
\newcommand{\massH}{\ensuremath{m_{\bosH}}}
\newcommand{\pT}{\ensuremath{p_{\mathrm{T}}}}
\newcommand{\mjj}{\ensuremath{m(jj)}}
\newcommand{\pTjj}{\ensuremath{\pT(jj)}}
\newcommand{\pTjmax}{\ensuremath{\pT(j)_{\mathrm{max}}}}
\newcommand{\pTjmin}{\ensuremath{\pT(j)_{\mathrm{min}}}}
\newcommand{\pTjadd}{\ensuremath{\pT(j)_{\mathrm{add}}}}
\newcommand{\btag}{\ensuremath{b\textrm{-}\mathrm{tag}}}
\newcommand{\btagmax}{\ensuremath{b\textrm{-}\mathrm{tag_{max}}}}
\newcommand{\btagmin}{\ensuremath{b\textrm{-}\mathrm{tag_{min}}}}
\newcommand{\btagadd}{\ensuremath{b\textrm{-}\mathrm{tag_{add}}}}
\newcommand{\pTV}{\ensuremath{\pT(V)}}
\newcommand{\pTmiss}{\ensuremath{p_{\mathrm{T}}^{\mathrm{miss}}}}
\newcommand{\pTmisstrk}{\ensuremath{p_{\mathrm{T}}^{\mathrm{miss, track}}}}

%These were some user commands I've run across that I thought some might want to incorporate into their work
%\newcommand{\bdm}{
 %   \begin{displaymath}}

%\newcommand{\edm}{
%    \end{displaymath}}

%\newcommand{\be}{
%    \begin{equation}}

%\newcommand{\ee}{
%    \end{equation}}

%\newcommand{\bea}{
 %   \begin{eqnarray}}

%\newcommand{\eea}{
%    \end{eqnarray}}
